% Created 2022-04-29 Fri 12:11
% Intended LaTeX compiler: pdflatex
\documentclass[t]{beamer}
\usepackage[utf8]{inputenc}
\usepackage[T1]{fontenc}
\usepackage{graphicx}
\usepackage{longtable}
\usepackage{wrapfig}
\usepackage{rotating}
\usepackage[normalem]{ulem}
\usepackage{amsmath}
\usepackage{amssymb}
\usepackage{capt-of}
\usepackage{hyperref}
\mode<beamer>{\usetheme{Amsterdam}}
\mode<beamer>{\usecolortheme{rose}}
\usepackage{fontspec}
\usepackage{polyglossia}
\setmainlanguage[babelshorthands=true]{german}
\usepackage{hyperref}
\usepackage{color}
\usepackage{xcolor}
\usepackage[misc]{ifsym}
\definecolor{darkblue}{rgb}{0,0,.5}
\definecolor{darkgreen}{rgb}{0,.5,0}
\definecolor{islamicgreen}{rgb}{0.0, 0.56, 0.0}
\definecolor{darkred}{rgb}{0.5,0,0}
\definecolor{mintedbg}{rgb}{0.95,0.95,0.95}
\definecolor{arsenic}{rgb}{0.23, 0.27, 0.29}
\definecolor{prussianblue}{rgb}{0.0, 0.19, 0.33}
\definecolor{coolblack}{rgb}{0.0, 0.18, 0.39}
\hypersetup{colorlinks=true, breaklinks=true, anchorcolor=blue,linkcolor=white, citecolor=islamicgreen, filecolor=darkred,  urlcolor=darkblue}
\usepackage{booktabs}
\usepackage{pgf}
\usepackage{minted}
\RequirePackage{fancyvrb}
\DefineVerbatimEnvironment{verbatim}{Verbatim}{fontsize=\scriptsize}
\usetheme{default}
\author{Göran Kirchner\thanks{e\_kirchnerg@doz.hwr-berlin.de}}
\date{2022-04-29}
\title{Funktionale Programmierung in F\# (4)}
\subtitle{Domain Driven Design \& Property Based Testing}
\hypersetup{
 pdfauthor={Göran Kirchner},
 pdftitle={Funktionale Programmierung in F\# (4)},
 pdfkeywords={},
 pdfsubject={},
 pdfcreator={Emacs 27.2 (Org mode 9.5.2)}, 
 pdflang={English}}
\begin{document}

\maketitle

\section{Ziel }
\label{sec:org531b5e9}
\begin{frame}[label={sec:orgffa894b}]{Programm}
\begin{itemize}
\item Hausaufgaben (4/7)
\begin{itemize}
\item[{$\boxtimes$}] Queen Attack
\item[{$\boxtimes$}] Raindrops
\item[{$\boxtimes$}] Gigaseconds
\item[{$\boxtimes$}] Bank Account
\end{itemize}
\item Domain Driven Design (DDD)
\item Property Based Testing
\end{itemize}
\end{frame}

\section{Hausaufgaben }
\label{sec:org27ad133}
\begin{frame}[label={sec:org5e5d2ba},fragile]{Queen Attack}
 \begin{minted}[bgcolor=mintedbg,frame=none,framesep=0pt,mathescape=true,fontsize=\scriptsize,breaklines=true,linenos=false,numbersep=5pt,gobble=0]{fsharp}
open System
let create (row, col) = row >= 0 && row < 8 && col >= 0 && col < 8
let canAttack (queen1: int * int) (queen2: int * int) = 
    let (r1, c1) = queen1
    let (r2, c2) = queen2
    Math.Abs(r1 - r2) = Math.Abs(c1 - c2) || r1 = r2 || c1 = c2
let whiteQueen1, blackQueen1 = (2, 2), (1, 1)
let test1 = canAttack blackQueen1 whiteQueen1
let whiteQueen2, blackQueen2 = (2, 4), (6, 6)
let test2 = canAttack blackQueen2 whiteQueen2
[test1; test2]
\end{minted}

\begin{verbatim}
val create: row: int * col: int -> bool
val canAttack: int * int -> int * int -> bool
val whiteQueen1: int * int = (2, 2)
val blackQueen1: int * int = (1, 1)
val test1: bool = true
val whiteQueen2: int * int = (2, 4)
val blackQueen2: int * int = (6, 6)
val test2: bool = false
val it: bool list = [true; false]
\end{verbatim}
\end{frame}

\begin{frame}[label={sec:orgff015bd},fragile]{Raindrops}
 \begin{minted}[bgcolor=mintedbg,frame=none,framesep=0pt,mathescape=true,fontsize=\scriptsize,breaklines=true,linenos=false,numbersep=5pt,gobble=0]{fsharp}
let rules =
    [ 3, "Pling"
      5, "Plang"
      7, "Plong" ]
let convert (number: int): string =
    let divBy n d = n % d = 0
    rules
    |> List.filter (fst >> divBy number)
    |> List.map snd
    |> String.concat ""
    |> function
       | "" -> string number
       | s -> s
let test = convert 105
test
\end{minted}

\begin{verbatim}
val rules: (int * string) list = [(3, "Pling"); (5, "Plang"); (7, "Plong")]
val convert: number: int -> string
val test: string = "PlingPlangPlong"
val it: string = "PlingPlangPlong"
\end{verbatim}
\end{frame}

\begin{frame}[label={sec:orgbaa4345},fragile]{Gigasecond}
 \begin{minted}[bgcolor=mintedbg,frame=none,framesep=0pt,mathescape=true,fontsize=\scriptsize,breaklines=true,linenos=false,numbersep=5pt,gobble=0]{fsharp}
let add (beginDate : System.DateTime) = beginDate.AddSeconds 1e9
let test = add (DateTime(2015, 1, 24, 22, 0, 0)) = (DateTime(2046, 10, 2, 23, 46, 40))
test
\end{minted}

\begin{verbatim}
val add: beginDate: DateTime -> DateTime
val test: bool = true
val it: bool = true
\end{verbatim}
\end{frame}

\begin{frame}[label={sec:orgd85fcd5},fragile]{Bank Account (1)}
 \begin{minted}[bgcolor=mintedbg,frame=none,framesep=0pt,mathescape=true,fontsize=\scriptsize,breaklines=true,linenos=false,numbersep=5pt,gobble=0]{fsharp}
type OpenAccount =
    { mutable Balance: decimal }
type Account =
    | Closed
    | Open of OpenAccount
let mkBankAccount() = Closed
let openAccount account =
    match account with
    | Closed -> Open { Balance = 0.0m }
    | Open _ -> failwith "Account is already open"
\end{minted}

\begin{verbatim}
type OpenAccount =
  { mutable Balance: decimal }
type Account =
  | Closed
  | Open of OpenAccount
val mkBankAccount: unit -> Account
val openAccount: account: Account -> Account
\end{verbatim}
\end{frame}

\begin{frame}[label={sec:org2a6fd98},fragile]{Bank Account (2)}
 \begin{minted}[bgcolor=mintedbg,frame=none,framesep=0pt,mathescape=true,fontsize=\scriptsize,breaklines=true,linenos=false,numbersep=5pt,gobble=0]{fsharp}
let closeAccount account =
    match account with
    | Open _ -> Closed
    | Closed -> failwith "Account is already closed"
let getBalance account =
    match account with
    | Open openAccount -> Some openAccount.Balance
    | Closed -> None
let updateBalance change account =
    match account with
    | Open openAccount ->
        lock (openAccount) (fun _ ->
            openAccount.Balance <- openAccount.Balance + change
            Open openAccount)
    | Closed -> failwith "Account is closed"
\end{minted}

\begin{verbatim}
val closeAccount: account: Account -> Account
val getBalance: account: Account -> decimal option
val updateBalance: change: decimal -> account: Account -> Account
\end{verbatim}
\end{frame}

\begin{frame}[label={sec:orga2d43dc},fragile]{Bank Account (3)}
 \begin{minted}[bgcolor=mintedbg,frame=none,framesep=0pt,mathescape=true,fontsize=\scriptsize,breaklines=true,linenos=false,numbersep=5pt,gobble=0]{fsharp}
let account = mkBankAccount() |> openAccount
let updateAccountAsync =        
    async { account |> updateBalance 1.0m |> ignore }
let ``updated from multiple threads`` =
    updateAccountAsync
    |> List.replicate 1000
    |> Async.Parallel 
    |> Async.RunSynchronously
    |> ignore
let test1 = getBalance account = (Some 1000.0m)
\end{minted}

\begin{verbatim}
val account: Account = Open { Balance = 1000.0M }
val updateAccountAsync: Async<unit>
\end{verbatim}
\end{frame}

\begin{frame}[label={sec:orgc90eee4}]{Pause}
\begin{block}{}
You’re bound to be unhappy if you optimize everything.

\null\hfill -- Donald Knuth
\end{block}
\end{frame}


\section{DDD (Domain Driven Design) }
\label{sec:org028e2fc}
\begin{frame}[label={sec:org11528d4}]{DDD}
\(\leadsto\) \href{./4.1 DDD\_With\_Fsharp.pdf}{Domain Driven Design}
\end{frame}

\begin{frame}[label={sec:org39fa7b2}]{Prinzipien}
\begin{itemize}
\item Verwende die Sprache der Domäne (ubiquitous Language)
\item Values und Entities
\item der Code ist das Design (kein UML nötig)
\item Design mit (algebraischen) Typen
\begin{itemize}
\item Option statt Null
\item DU statt Vererbung
\end{itemize}
\item illegale Konstellationen sollten nicht repräsentierbar sein!
\end{itemize}
\end{frame}

\begin{frame}[label={sec:orgf55ebf9}]{Pause}
\begin{block}{}
Are you quite sure that all those bells and whistles, all those wonderful facilities of your so called powerful programming languages, belong to the solution set rather than the problem set?

\null\hfill -- Edsger Dijkstra
\end{block}
\end{frame}

\begin{frame}[label={sec:orgf747148}]{DDD Übung 1 (Contacts -- ex 2)}
A Contact has

\begin{itemize}
\item a personal name
\item an optional email address
\item an optional postal address
\item Rule: a contact must have an email or a postal address
\end{itemize}

A Personal Name consists of a first name, middle initial, last name

\begin{itemize}
\item Rule: the first name and last name are required
\item Rule: the middle initial is optional
\item Rule: the first name and last name must not be more than 50 chars
\item Rule: the middle initial is exactly 1 char, if present
\end{itemize}

A postal address consists of a four address fields plus a country

\begin{itemize}
\item Rule: An Email Address can be verified or unverified
\end{itemize}
\end{frame}

\begin{frame}[label={sec:orgd8845cf}]{DDD Übung 2 (Payments -- ex 3)}
The payment taking system should accept:

\begin{itemize}
\item Cash
\item Credit cards
\item Cheques
\item Paypal
\item Bitcoin
\end{itemize}

A payment consists of a:

\begin{itemize}
\item payment
\item non-negative amount
\end{itemize}

After designing the types, create functions that will:

\begin{itemize}
\item print a payment method
\item print a payment, including the amount
\item create a new payment from an amount and method
\end{itemize}
\end{frame}

\begin{frame}[label={sec:org41669ae}]{DDD Übung 3 (Refactoring -- ex 4)}
Much C\# code has implicit states that you can recognize by fields called "IsSomething", or nullable date.

This is a sign that states transitions are present but not being modelled properly.
\end{frame}

\begin{frame}[label={sec:orge2b41b8}]{DDD Übung 4 (Shopping Cart -- fsm ex 3)}
Create types that model an e-commerce shopping cart.

\begin{itemize}
\item Rule: "You can't remove an item from an empty cart"!
\item Rule: "You can't change a paid cart"!
\item Rule: "You can't pay for a cart twice"!
\end{itemize}

States are:
\begin{itemize}
\item Empty
\item ActiveCartData
\item PaidCartData
\end{itemize}
\end{frame}

\begin{frame}[label={sec:org09a213c}]{Pause}
\begin{block}{}
About the use of language: it is impossible to sharpen a pencil with a blunt axe. 
It is equally vain to try to do it with ten blunt axes instead.

\null\hfill -- Edsger Dijkstra
\end{block}
\end{frame}

\section{Property Based Testing }
\label{sec:orga735bb0}
\begin{frame}[label={sec:orgd7da06d},fragile]{Example Based Tests :)}
 \begin{minted}[bgcolor=mintedbg,frame=none,framesep=0pt,mathescape=true,fontsize=\scriptsize,breaklines=true,linenos=false,numbersep=5pt,gobble=0]{fsharp}
open System
#r "../src/4/02_PBT/lib/Expecto.dll"
open Expecto
let expectoConfig = {defaultConfig with colour = Expecto.Logging.ColourLevel.Colour0}
#load "../src/4/02_PBT/A1_Add_Implementations.fsx"
open A1_Add_Implementations
\end{minted}

\begin{minted}[bgcolor=mintedbg,frame=none,framesep=0pt,mathescape=true,fontsize=\scriptsize,breaklines=true,linenos=false,numbersep=5pt,gobble=0]{fsharp}
module Test1 =
    open Implementation1
    let tests = testList "implementation 1" [
        test "add 1 3 = 4" {
            let actual = add 1 3
            let expected = 4
            Expect.equal expected actual "" }
        test "add 2 2 = 4" {
            let actual = add 2 2
            let expected = 4
            Expect.equal expected actual "" } ];;
runTests expectoConfig Test1.tests
\end{minted}

\begin{verbatim}
[23:23:18 INF] EXPECTO? Running tests... <Expecto>
[23:23:18 INF] EXPECTO! 2 tests run in 00:00:00.0117930 for implementation 1 – 2 passed, 0 ignored, 0 failed, 0 errored. Success! <Expecto>
val it: int = 0
\end{verbatim}
\end{frame}

\begin{frame}[label={sec:org73301b5},fragile]{Evil Developer From Hell :(}
 \begin{minted}[bgcolor=mintedbg,frame=none,framesep=0pt,mathescape=true,fontsize=\scriptsize,breaklines=true,linenos=false,numbersep=5pt,gobble=0]{fsharp}
module Implementation1 =
    let add x y =
        4
\end{minted}

\begin{verbatim}
module Implementation1 =
  val add: x: 'a -> y: 'b -> int
\end{verbatim}
\end{frame}

\begin{frame}[label={sec:org24d1c2f}]{PBT}
\(\leadsto\) \href{./4.2 An introduction to property based testing.pdf}{Property Based Testing}
\end{frame}

\begin{frame}[label={sec:orgb14df37},fragile]{FsCheck}
 \begin{minted}[bgcolor=mintedbg,frame=none,framesep=0pt,mathescape=true,fontsize=\scriptsize,breaklines=true,linenos=false,numbersep=5pt,gobble=0]{fsharp}
open System
#r "../src/4/02_PBT/lib/FsCheck.dll"
\end{minted}

\begin{minted}[bgcolor=mintedbg,frame=none,framesep=0pt,mathescape=true,fontsize=\scriptsize,breaklines=true,linenos=false,numbersep=5pt,gobble=0]{fsharp}
let add1 x y = x + y
let add2 x y = x - y
let commutativeProperty f x y =
   let result1 = f x y
   let result2 = f y x
   result1 = result2;;
FsCheck.Check.Quick (commutativeProperty add1)
FsCheck.Check.Quick (commutativeProperty add2)
\end{minted}

\begin{verbatim}
Ok, passed 100 tests.
Falsifiable, after 1 test (0 shrinks) (StdGen (501447179, 297030377)):
Original:
0
1
val it: unit = ()
\end{verbatim}
\end{frame}

\begin{frame}[label={sec:orgdea4c61},fragile]{FsCheck (Generate)}
 \begin{minted}[bgcolor=mintedbg,frame=none,framesep=0pt,mathescape=true,fontsize=\scriptsize,breaklines=true,linenos=false,numbersep=5pt,gobble=0]{fsharp}
type Temp = F of int | C of float;;
let fGen =
    FsCheck.Gen.choose(32,212)
    |> FsCheck.Gen.map (fun i -> F i);;
let cGen =
    FsCheck.Gen.choose(0,100)
    |> FsCheck.Gen.map (fun i -> C (float i));;
let tempGen =
    FsCheck.Gen.oneof [fGen; cGen]

let test = tempGen |> FsCheck.Gen.sample 0 20
test
\end{minted}

\begin{verbatim}
val tempGen: FsCheck.Gen<Temp> = Gen <fun:Bind@88>
val test: Temp list =
  [C 86.0; C 5.0; F 53; F 211; F 179; F 156; F 124; F 101; C 62.0; F 136;
   C 75.0; F 32; C 25.0; F 158; C 60.0; F 199; C 89.0; F 124; F 34; C 71.0]
val it: Temp list =
  [C 86.0; C 5.0; F 53; F 211; F 179; F 156; F 124; F 101; C 62.0; F 136;
   C 75.0; F 32; C 25.0; F 158; C 60.0; F 199; C 89.0; F 124; F 34; C 71.0]
\end{verbatim}
\end{frame}

\begin{frame}[label={sec:orge1eb870},fragile]{FsCheck (Shrink)}
 \begin{minted}[bgcolor=mintedbg,frame=none,framesep=0pt,mathescape=true,fontsize=\scriptsize,breaklines=true,linenos=false,numbersep=5pt,gobble=0]{fsharp}
open FsCheck
let smallerThan81Property x = x < 81
FsCheck.Check.Quick smallerThan81Property

let test1 = FsCheck.Arb.shrink 100 |> Seq.toList
let test2 = FsCheck.Arb.shrink 88 |> Seq.toList
test2
\end{minted}

\begin{verbatim}
Ok, passed 100 tests.
val smallerThan81Property: x: int -> bool
val test1: int list = [0; 50; 75; 88; 94; 97; 99]
val test2: int list = [0; 44; 66; 77; 83; 86; 87]
val it: int list = [0; 44; 66; 77; 83; 86; 87]
\end{verbatim}
\end{frame}

\begin{frame}[label={sec:org28fd5b5}]{Auswahl der Eigenschaften}
\begin{itemize}
\item Unterschiedlicher Weg, gleiches Ziel (Map(f)(Option(x))=Option(f x))
\item Hin und Her (z.B. Reverse einer Liste)
\item Invarianten (z.B. Länge einer Liste bei Sortierung)
\item Idempotenz (noch einmal bringt nichts mehr)
\item Divide et Impera! (teile und herrsche)
\item Hard to prove, easy to verify (Primzahlzerlegung)
\item Test-Orakel (z.B. einfach aber langsam)
\end{itemize}
\end{frame}

\section{Ende }
\label{sec:orgf7da6b3}
\begin{frame}[label={sec:org1913c15}]{Zusammenfassung}
\begin{itemize}
\item funktionales Domain Modeling (DDD)
\item eigenschaftsbasiertes Testen (Property Based Testing)
\end{itemize}
\end{frame}

\begin{frame}[label={sec:orge6d5d17}]{Links}
\begin{itemize}
\item \href{https://fsharpforfunandprofit.com/ddd/}{Domain Driven Design}
\item \href{https://fsharpforfunandprofit.com/books/}{Domain Modeling Made Functional}
\item \href{https://github.com/fscheck/FsCheck}{FsCheck}
\item \href{https://fsharpforfunandprofit.com/posts/property-based-testing/}{An introduction to property-based testing}
\item \href{https://fsharpforfunandprofit.com/posts/property-based-testing-2/}{Choosing properties for property-based testing}
\end{itemize}
\end{frame}
\end{document}