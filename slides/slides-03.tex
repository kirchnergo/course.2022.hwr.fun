% Created 2022-04-07 Thu 11:22
% Intended LaTeX compiler: pdflatex
\documentclass[t]{beamer}
\usepackage[utf8]{inputenc}
\usepackage[T1]{fontenc}
\usepackage{graphicx}
\usepackage{longtable}
\usepackage{wrapfig}
\usepackage{rotating}
\usepackage[normalem]{ulem}
\usepackage{amsmath}
\usepackage{amssymb}
\usepackage{capt-of}
\usepackage{hyperref}
\mode<beamer>{\usetheme{Amsterdam}}
\mode<beamer>{\usecolortheme{rose}}
\usepackage{fontspec}
\usepackage{polyglossia}
\setmainlanguage[babelshorthands=true]{german}
\usepackage{hyperref}
\usepackage{color}
\usepackage{xcolor}
\usepackage[misc]{ifsym}
\definecolor{darkblue}{rgb}{0,0,.5}
\definecolor{darkgreen}{rgb}{0,.5,0}
\definecolor{islamicgreen}{rgb}{0.0, 0.56, 0.0}
\definecolor{darkred}{rgb}{0.5,0,0}
\definecolor{mintedbg}{rgb}{0.95,0.95,0.95}
\definecolor{arsenic}{rgb}{0.23, 0.27, 0.29}
\definecolor{prussianblue}{rgb}{0.0, 0.19, 0.33}
\definecolor{coolblack}{rgb}{0.0, 0.18, 0.39}
\hypersetup{colorlinks=true, breaklinks=true, anchorcolor=blue,linkcolor=white, citecolor=islamicgreen, filecolor=darkred,  urlcolor=darkblue}
\usepackage{booktabs}
\usepackage{pgf}
\usepackage{minted}
\RequirePackage{fancyvrb}
\DefineVerbatimEnvironment{verbatim}{Verbatim}{fontsize=\scriptsize}
\usetheme{default}
\author{Göran Kirchner\thanks{e\_kirchnerg@doz.hwr-berlin.de}}
\date{2022-04-08}
\title{Funktionale Programmierung in F\# (3)}
\subtitle{Grundlagen \& Funktionales Design}
\hypersetup{
 pdfauthor={Göran Kirchner},
 pdftitle={Funktionale Programmierung in F\# (3)},
 pdfkeywords={},
 pdfsubject={},
 pdfcreator={Emacs 27.2 (Org mode 9.5.2)}, 
 pdflang={English}}
\begin{document}

\maketitle

\section{Ziel }
\label{sec:orga24e0f7}

\begin{frame}[label={sec:orgb00a92c}]{Programm}
\begin{itemize}
\item Vertiefung Railway-Oriented Programming
\item Prinzipien des funktionalen Designs
\item Refactoring (Übung)
\end{itemize}
\end{frame}

\section{Railway-Oriented Programming (Wdh.) }
\label{sec:org884c213}

\begin{frame}[label={sec:org3a4f635},fragile]{Übung 1}
 \begin{itemize}
\item Implementiere einen Workflow (\texttt{validateInput}).
\end{itemize}

\begin{minted}[bgcolor=mintedbg,frame=none,framesep=0pt,mathescape=true,fontsize=\scriptsize,breaklines=true,linenos=false,numbersep=5pt,gobble=0]{fsharp}
type Input = {Name : string; Email : string }
let checkNameNotBlank input =
  if input.Name = "" then
     Error "Name must not be blank"
  else Ok input
let checkName50 input =
  if input.Name.Length > 50 then
     Error "Name must not be longer than 50 chars"
  else Ok input
let checkEmailNotBlank input =
  if input.Email = "" then
     Error "Email must not be blank"
  else Ok input
\end{minted}

\begin{verbatim}
type Input =
  {
    Name: string
    Email: string
  }
val checkNameNotBlank: input: Input -> Result<Input,string>
val checkName50: input: Input -> Result<Input,string>
val checkEmailNotBlank: input: Input -> Result<Input,string>
\end{verbatim}
\end{frame}

\begin{frame}[label={sec:orgd1da6cc},fragile]{Übung 1 (Lösung)}
 \begin{minted}[bgcolor=mintedbg,frame=none,framesep=0pt,mathescape=true,fontsize=\scriptsize,breaklines=true,linenos=false,numbersep=5pt,gobble=0]{fsharp}
let validateInput input =
    input
    |> checkNameNotBlank
    |> Result.bind checkName50
    |> Result.bind checkEmailNotBlank

let goodInput = {Name="Max"; Email="x@example.com"}
let blankName = {Name=""; Email="x@example.com"}
let blankEmail = {Name="Nora"; Email=""}
[validateInput goodInput; validateInput blankName; validateInput blankEmail]
\end{minted}

\begin{verbatim}
val validateInput: input: Input -> Result<Input,string>
val goodInput: Input = { Name = "Max"
                         Email = "x@example.com" }
val blankName: Input = { Name = ""
                         Email = "x@example.com" }
val blankEmail: Input = { Name = "Nora"
                          Email = "" }
val it: Result<Input,string> list =
  [Ok { Name = "Max"
        Email = "x@example.com" }; Error "Name must not be blank";
   Error "Email must not be blank"]
\end{verbatim}
\end{frame}

\begin{frame}[label={sec:org550180a},fragile]{Übung 2}
 \begin{itemize}
\item Definiere einen \emph{Custom Error Type}. Benutze diesen in den Validierungen.
\item Übersetze die Fehlermeldungen (EN, FR, DE?).
\end{itemize}

\begin{minted}[bgcolor=mintedbg,frame=none,framesep=0pt,mathescape=true,fontsize=\scriptsize,breaklines=true,linenos=false,numbersep=5pt,gobble=0]{fsharp}
type ErrorMessage =
  | ??   // name not blank
  | ?? of int  // name not longer than
  | ??   // email not longer than
let translateError_EN err =
  match err with
  | ?? -> "Name must not be blank"
  | ?? i -> sprintf "Name must not be longer than %i chars" i
  | ?? -> "Email must not be blank"
  | SmtpServerError msg -> sprintf "SmtpServerError [%s]" msg
\end{minted}
\end{frame}

\begin{frame}[label={sec:orgd86303b},fragile]{Übung 2 (Lösung)}
 \begin{minted}[bgcolor=mintedbg,frame=none,framesep=0pt,mathescape=true,fontsize=\scriptsize,breaklines=true,linenos=false,numbersep=5pt,gobble=0]{fsharp}
type ErrorMessage =
    | NameMustNotBeBlank
    | NameMustNotBeLongerThan of int
    | EmailMustNotBeBlank
    | SmtpServerError of string
let translateError_FR err =
    match err with
    | NameMustNotBeBlank -> "Nom ne doit pas être vide"
    | NameMustNotBeLongerThan i -> sprintf "Nom ne doit pas être plus long que %i caractères" i
    | EmailMustNotBeBlank -> "Email doit pas être vide"
    | SmtpServerError msg -> sprintf "SmtpServerError [%s]" msg
\end{minted}
\end{frame}

\section{Prinzipien des Funktionalen Designs }
\label{sec:orgdeacdc7}
\begin{frame}[label={sec:orgf536af6}]{Funktionales Design}
\(\leadsto\) \href{./3.1 Functional Design Patterns.pdf}{Functional Design Patterns}
\end{frame}

\begin{frame}[label={sec:orgcb5b294}]{Prinzipien (1)}
\begin{itemize}
\item Funktionen sind Daten!
\item überall Verkettung (Composition)
\item überall Funktionen
\item Typen sind keine Klassen
\item Typen kann man ebenfalls verknüpfen (algebraische Datentypen)
\item Typsignaturen lügen nicht!
\item statische Typen zur Modellierung der Domäne (später mehr;)
\end{itemize}
\end{frame}

\begin{frame}[label={sec:org43f3a35}]{Prinzipien (2)}
\begin{itemize}
\item Parametrisiere alles!
\item Typsignaturen sind "Interfaces"
\item Partielle Anwendung ist "Dependency Injection"
\item Monaden entsprechen dem "Chaining of Continuations"
\begin{itemize}
\item bind für Options
\item bind für Fehler
\item bind für Tasks
\end{itemize}
\item "map" - Funktionen
\begin{itemize}
\item Nutze "map" - Funktion von generische Typen!
\item wenn man einen generischen Typ definiert, dann auch eine "map" - Funktion
\end{itemize}
\end{itemize}
\end{frame}

\begin{frame}[label={sec:org7ebf968}]{Übung 3}
\begin{itemize}
\item Typsignaturen
\item Funktionen sind Daten
\end{itemize}
\end{frame}

\begin{frame}[label={sec:org5e27385},fragile]{Übung 4 (Think of a Number)}
 \begin{minted}[bgcolor=mintedbg,frame=none,framesep=0pt,mathescape=true,fontsize=\scriptsize,breaklines=true,linenos=false,numbersep=5pt,gobble=0]{fsharp}
let thinkOfANumber numberYouThoughtOf =
    let addOne x = x + 1
    let squareIt x = ??
    let subtractOne x = ??
    let divideByTheNumberYouFirstThoughtOf x = ??
    let subtractTheNumberYouFirstThoughtOf x = ??

    // define these functions
    // then combine them using piping

    numberYouThoughtOf
    |> ??
    |> ??
    |> ??
\end{minted}
\end{frame}

\begin{frame}[label={sec:org5af5f2d},fragile]{Übung 4 (Lösung)}
 \begin{minted}[bgcolor=mintedbg,frame=none,framesep=0pt,mathescape=true,fontsize=\scriptsize,breaklines=true,linenos=false,numbersep=5pt,gobble=0]{fsharp}
let thinkOfANumber numberYouThoughtOf =
    let addOne x = x + 1
    let squareIt x = x * x
    let subtractOne x = x - 1
    let divideByTheNumberYouFirstThoughtOf x = x / numberYouThoughtOf
    let subtractTheNumberYouFirstThoughtOf x = x - numberYouThoughtOf
    numberYouThoughtOf
    |> addOne
    |> squareIt
    |> subtractOne
    |> divideByTheNumberYouFirstThoughtOf
    |> subtractTheNumberYouFirstThoughtOf
thinkOfANumber 42
\end{minted}

\begin{verbatim}
val thinkOfANumber: numberYouThoughtOf: int -> int
val it: int = 2
\end{verbatim}
\end{frame}

\begin{frame}[label={sec:org5aefc41},fragile]{Übung 5 (Decorator)}
 \begin{itemize}
\item Implementiere das \href{https://de.wikipedia.org/wiki/Decorator}{Decorator-Emtwurfsmuster} für \texttt{add1}.
\end{itemize}
\end{frame}

\begin{frame}[label={sec:orgf8a9d2e}]{Pause}
\begin{block}{}
If we’d asked the customers what they wanted, they would have said “faster horses”.

\null\hfill -- Henry Ford
\end{block}
\end{frame}

\section{Refactoring }
\label{sec:org2e09a26}
\begin{frame}[label={sec:org917ad1c},fragile]{Tree Building (Übung)}
 \begin{minted}[bgcolor=mintedbg,frame=none,framesep=0pt,mathescape=true,fontsize=\scriptsize,breaklines=true,linenos=false,numbersep=5pt,gobble=0]{shell}
exercism download --exercise=tree-building --track=fsharp
\end{minted}
\end{frame}

\begin{frame}[label={sec:orgd78e686},fragile]{Tree Building (Imperativ)}
 \begin{minted}[bgcolor=mintedbg,frame=none,framesep=0pt,mathescape=true,fontsize=\scriptsize,breaklines=true,linenos=false,numbersep=5pt,gobble=0]{fsharp}
let buildTree records =
    let records' = List.sortBy (fun x -> x.RecordId) records
    if List.isEmpty records' then failwith "Empty input"
    else
        let root = records'.[0]
        if (root.ParentId = 0 |> not) then
            failwith "Root node is invalid"
        else
            if (root.RecordId = 0 |> not) then failwith "Root node is invalid"
            else
                let mutable prev = -1
                let mutable leafs = []
                for r in records' do
                    if (r.RecordId <> 0 && (r.ParentId > r.RecordId || r.ParentId = r.RecordId)) then
                        failwith "Nodes with invalid parents"
                    else
                        if r.RecordId <> prev + 1 then
                            failwith "Non-continuous list"
                        else
                            prev <- r.RecordId
                            if (r.RecordId = 0) then
                                leafs <- leafs @ [(-1, r.RecordId)]
                            else
                                leafs <- leafs @ [(r.ParentId, r.RecordId)]
\end{minted}
\end{frame}

\begin{frame}[label={sec:orga0b4e86},fragile]{Tree Building (Funktional)}
 \begin{minted}[bgcolor=mintedbg,frame=none,framesep=0pt,mathescape=true,fontsize=\scriptsize,breaklines=true,linenos=false,numbersep=5pt,gobble=0]{fsharp}
let buildTree records = 
    records
    |> List.sortBy (fun r -> r.RecordId)
    |> validate
    |> List.tail
    |> List.groupBy (fun r -> r.ParentId)
    |> Map.ofList
    |> makeTree 0

let rec makeTree id map =
    match map |> Map.tryFind id with
    | None -> Leaf id
    | Some list -> Branch (id, 
        list |> List.map (fun r -> makeTree r.RecordId map))
\end{minted}
\end{frame}

\begin{frame}[label={sec:org4876361},fragile]{Tree Building (Error Handling)}
 \begin{minted}[bgcolor=mintedbg,frame=none,framesep=0pt,mathescape=true,fontsize=\scriptsize,breaklines=true,linenos=false,numbersep=5pt,gobble=0]{fsharp}
let validate records =
    match records with
    | [] -> failwith "Input must be non-empty"
    | x :: _ when x.RecordId <> 0 -> 
        failwith "Root must have id 0"
    | x :: _ when x.ParentId <> 0 -> 
        failwith "Root node must have parent id 0"
    | _ :: xs when xs |> List.exists (fun r -> r.RecordId < r.ParentId) -> 
        failwith "ParentId should be less than RecordId"
    | _ :: xs when xs |> List.exists (fun r -> r.RecordId = r.ParentId) -> 
        failwith "ParentId cannot be the RecordId except for the root node."
    | rs when (rs |> List.map (fun r -> r.RecordId) |> List.max) > (List.length rs - 1) -> 
        failwith "Ids must be continuous"
    | _ -> records
\end{minted}
\end{frame}

\begin{frame}[label={sec:orga0da274},fragile]{Tree Building (Benchmarking)}
 \begin{itemize}
\item \href{https://github.com/dotnet/BenchmarkDotNet}{BenchmarkDotNet}
\end{itemize}

\begin{minted}[bgcolor=mintedbg,frame=none,framesep=0pt,mathescape=true,fontsize=\scriptsize,breaklines=true,linenos=false,numbersep=5pt,gobble=0]{shell}
dotnet run -c release
\end{minted}

\begin{minted}[bgcolor=mintedbg,frame=none,framesep=0pt,mathescape=true,fontsize=\scriptsize,breaklines=true,linenos=false,numbersep=5pt,gobble=0]{shell}
sed -n 447,460p $benchmarks
\end{minted}

\tiny
// * Summary *

BenchmarkDotNet=v0.12.1, OS=macOS 12.3 (21E230) [Darwin 21.4.0]
Intel Core i7-7920HQ CPU 3.10GHz (Kaby Lake), 1 CPU, 8 logical and 4 physical cores
.NET Core SDK=6.0.200
  [Host]     : .NET Core 6.0.2 (CoreCLR 6.0.222.6406, CoreFX 6.0.222.6406), X64 RyuJIT DEBUG
  DefaultJob : .NET Core 6.0.2 (CoreCLR 6.0.222.6406, CoreFX 6.0.222.6406), X64 RyuJIT


\begin{center}
\begin{tabular}{lllllrrrlll}
Method & Mean & Error & StdDev & Median & Ratio & RatioSD & Gen 0 & Gen 1 & Gen 2 & Allocated\\
\hline
Baseline & 8.058 μs & 0.1562 μs & 0.1535 μs & 8.069 μs & 1.00 & 0.00 & 3.3569 & - & - & 13.75 KB\\
Mine & 5.172 μs & 0.2075 μs & 0.5953 μs & 5.006 μs & 0.57 & 0.02 & 1.8768 & - & - & 7.68 KB\\
\end{tabular}
\end{center}
\end{frame}


\section{Ende }
\label{sec:orge3bd3d3}
\begin{frame}[label={sec:org0e30a04}]{Zusammenfassung}
\begin{itemize}
\item funktionaler Umgang mit Fehlern (ROP)
\item funktionales Design
\item funktionales Refactoring
\end{itemize}
\end{frame}

\begin{frame}[label={sec:org29f2505}]{Links}
\begin{itemize}
\item \href{https://fsharp.org/}{fsharp.org}
\item \href{https://docs.microsoft.com/de-de/dotnet/fsharp/}{docs.microsoft.com/../dotnet/fsharp}
\item \href{https://sergeytihon.com/}{F\# weekly}
\item \href{https://fsharpforfunandprofit.com/}{fsharpforfunandprofit.com}
\item \href{https://github.com/fsprojects/awesome-fsharp}{github.com/../awesome-fsharp}
\end{itemize}
\end{frame}

\begin{frame}[label={sec:org7a628ce}]{Hausaufgabe (Erinnerung)}
\begin{itemize}
\item exercism.io (E-Mail bis 22.04)
\begin{itemize}
\item[{$\square$}] Queen Attack
\item[{$\square$}] Raindrops
\item[{$\square$}] Gigasecond
\item[{$\square$}] Bank Account
\item[{$\square$}] Accumulate
\item[{$\square$}] Space Age
\end{itemize}

\item exercism.io (E-Mail bis 29.04)
\begin{itemize}
\item[{$\square$}] Poker (Programmieraufgabe)
\end{itemize}
\end{itemize}
\end{frame}
\end{document}